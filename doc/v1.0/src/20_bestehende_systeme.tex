Nachfolgend werden einige bestehende und etablierte Monitoring Systeme\index{Monitoring!System} betrachtet und kurz zusammengefasst. Die Analyse soll deren St\"arken wie auch Schw\"achen aufzeigen und dazu dienen, m\"ogliche Funktionen und Techniken f\"ur das zu entwickelnde System ausfindig zu machen.

Ein weiterer Schwerpunkt der Analyse ist es, zu schauen, inwieweit das jeweilige System sich als hochverf\"ugbare Monitoring-Instanz nutzen l\"asst. Dies setzt voraus, dass es nicht nur eine einzelne Instanz gibt, welche die \"uberwachung ausf\"uhrt, sondern mehrere. So kann gew\"ahrleistet werden, dass wenn eine Instanz ausf\"allt, die \"Uberwachung dennoch statfindet. Ein System darf zudem nicht von zu vielen Umsystemen abh\"angig sein, denn je mehr Abh\"angigkeiten ein Produkt mit sich bringt, desto mehr Fehlerquellen und somit Fehlermeldungen und nicht \"uberwachte Systeme kann es geben.

%%%%%%%%%%%%%%%%%%%%%%%%%%%%%%%%%%%%%%%%%%%%%%%%%%%%%%%%%%%%%%%%%%%%%%%%%%%%%%%%%%%%%%%%%%%%%%%%%%%%%%%%%%%%%%%%%%%%%%%%%%%%%%%%%%%%%%%%%
\section{Systeme im \"Uberblick} \label{sec:systeme}
\subsection{Cacti} \label{sec:systeme-cacti} \index{Cacti} \index{Monitoring!Cacti}
  Cacti\cite{cacti} ist ein Monitoring-Tool basierend auf dem RRDtool\footnote{\label{foot:rrdtools}RRDtool\cite{rrdtool} - \url{http://oss.oetiker.ch/rrdtool/} - ist ein Quasi-Standard in der OpenSource Gemeinde f\"ur high performance Logging und Visualisierung. Vorteil des RRD-Formates ist es, dass die Datenbank grunds\"atzlich nicht \"uberlaufen kann.} von Tobias Oetiker. Frontend und Backend sind in PHP\footnote{\label{foot:php}\url{http://www.php.net/}} geschrieben, wobei die Konfiguration in einer MySQL-Datenbank\footnote{\label{foot:mysql}\url{http://www.mysql.com/}} gespeichert wird. Zur Konfiguration der Endger\"ate und der Graphen stehen viele Templates zur Verf\"ugung, welche individuell angepasst werden k\"onnen.

  Damit man auch Kunden ihre eigenen \"uberwachten Systeme anzeigen kann, besitzt Cacti die M\"oglichkeit Benutzer und Gruppen anzulegen sowie diese auf verschiedene Bereiche und Graphen einzuschr\"anken.

  Nebst RRDtool setzt Cacti einen funktionsf\"ahigen Webserver mit PHP inklusive net-snmp sowie eine MySQL-Datenbank voraus. Sind diese Voraussetzungen gegeben, kann Cacti direkt in ein Verzeichnis auf dem Webserver kopiert und anschliessend \"uber eine Weboberfl\"ache konfiguriert werden. F\"ur die regelm\"assige Pr\"ufung aller konfigurierten Systeme muss nach der Konfiguration noch ein Cron-Job\footnote{Unter Windows ein \textit{Geplanter Task}} eingerichtet werden, welcher alle 5 Minuten eine PHP-Datei ausf\"uhrt.

  Zus\"atzliche zu \"uberwachende Systeme k\"onnen mittels der Weboberfl\"ache eingerichtet und in einer Baumstruktur abgelegt werden. Zur einfacheren Konfiguration stehen diverse Templates f\"ur Devices, Betriebssysteme und Graphen zur Verf\"ugung. Diese Templates k\"onnen angepasst und individualisiert werden. Dies ist f\"ur einen ge\"ubten Administrator zwar von Vorteil, jedoch kann die ganze Angelegenheit auch schnell sehr komplex und un\"ubersichtlich werden.

  Die Graphen werden dem Benutzer in einer Baumstruktur angezeigt. Diese weisen (je nach verwendetem Daten- und Graphtemplate) noch zus\"atzliche Informationen wie Minimal-, Maximal-, Durchschnitts- oder andere Werte auf. Einzelne Bereiche k\"onnen mit der Maus markiert und dann genauer betrachtet werden. Dadurch k\"onnen auch \"altere Ereignisse im Nachhinein genauer analysiert werden.

  Cacti bietet keine automatische Alerting-Funktionalit\"at, diese muss \"uber Plugins wie \texttt{monitor}\footnote{Das Monitor-Plugin zeigt alle Hosts in einem Raster an. F\"allt ein Host aus, wird er hervorgehoben und ein Alarm-Sound erklingt.} von \url{http://cactiusers.org/} nachtr\"aglich installiert werden.

\subsubsection{Fazit} \label{sec:systeme-cacti-fazit}
  Das Haupt-Einsatzgebiet von Cacti liegt in der \"Uberwachung von SNMP\index{SNMP}-f\"ahigen Ger\"aten. Dies k\"onnen sowohl Serversysteme sein, bei welchen Ressourcen \"uberwacht werden sollen, aber auch Router und andere Ger\"ate die Trafficdaten per SNMP anbieten. Grunds\"atzlich kann Cacti sehr gut auf einem bestehenden Webserver installiert werden, denn die regelm\"assigen Pr\"ufungen sind nicht sehr Performance lastig (je nach dem wie viele Systeme \"uberpr\"uft werden).

  Durch das fehlende Alerting ist Cacti weniger geeignet f\"ur hochverf\"ugbare Systeme, denn die Graphen m\"ussen visuell \"uberwacht werden. Cacti hat seine St\"arken in der Verwaltung und Darstellung der Daten sowie der M\"oglichkeiten zur Datenabfrage und -Aufbereitung (individuelle SNMP-OIDs, etc.).


\subsection{SmokePing} \label{sec:systeme-smoke} \index{SmokePing} \index{Monitoring!SmokePing}
  SmokePing\cite{smokeping} ist ein auf dem RRDtool\cite{rrdtool} basierendes Tool zur \"Uberwachung der Netzwerk-Latenzen. Wie das RRDtool ist auch SmokePing von Tobi Oetiker entwickelt. Das komplette Front- und Backend von SmokePing ist in Perl geschrieben und setzt neben einem Webserver\footnote{\label{foot:apache-suexec}Der Webserver muss CGI f\"ahig sein. Pr\"aferiert wird ein Apache mit der Option SuExec, welche es erlaubt CGI-Skripte als normalen Benutzer auszuf\"uhren.} und einigen Perl-Modulen\footnote{Perl-Module f\"ur SmokePing: LWP::UserAgent, CGI::Carp} nur FPing\cite{fping} voraus. Neben Latenz-Tests k\"onnen mit SmokePing auch Performance-\footnote{F\"ur Performance-Tests muss EchoPing vorhanden sein: \url{http://www.echoping.sf.net/}}, DNS-\footnote{F\"ur DNS-Tests muss das Perl Modul Net::DNS sowie das Kommando \texttt{dig} vorhanden sein: \url{http://www.isc.org/bind/}}, SSH-\footnote{F\"ur SSH-Tests muss das Perl Modul IO::Soket::SSL sowie OpenSSH vorhanden sein: \url{http://www.openssh.org/}}, Radius-\footnote{F\"ur Radius-Tests muss das Perl Modul Authen::Radius vorhanden sein}, LDAP-\footnote{F\"ur LDAP-Tests muss das Perl Modul Net::LDAP vorhanden sein} und andere Tests durchgef\"uhrt werden.

  SmokePing muss, wenn alle Voraussetzungen erf\"ullt sind, lediglich in ein Verzeichnis auf dem Server kopiert werden. Dieses Verzeichnis muss anschliessend im Webserver als CGI-Verzeichnis konfiguriert werden. F\"ur die \"Uberwachung muss der SmokePing-Daemon noch so eingerichtet werden, dass dieser auch nach einem Neustart wieder automatisch gestartet wird. Damit der SmokePing-Daemon die Daten richtig speichert und die korrekten Dateien erzeugt, m\"ussen zwei Perl-Dateien angepasst und \"uberpr\"uft werden.

  SmokePing wird grunds\"atzlich \"uber eine einzelne Datei konfiguriert. Dabei werden die zu \"uberwachenden Systeme mittels einer speziellen Notation in einer Baumstruktur angelegt und parametriert. Jedes System kann mit unterschiedlichen \"Uberwachungen und Alerting-Regeln belegt werden.

  Die Graphen werden in der durch die Konfiguration festgelegten Baumstruktur dargestellt. Die einzelnen \"Asten enthalten die jeweiligen \"Ubersichtsgraphen, w\"ahrend die Bl\"atter alle konfigurierten Graphen mit den verschiedenen Zeitintervallen zeigen. Zur nachtr\"aglichen Pr\"ufung von Ereignissen k\"onnen Bereiche in einem Graphen markiert und vergr\"ossert dargestellt werden.

  F\"allt ein System aus, wird das Alerting von SmokePing in Betrieb gesetzt. Dieses kann pro \"uberwachtem System durch mehrere Alerting-Mechanismen individuell gestaltet werden. So kann zum Beispiel definiert werden, dass bei einem Packet-Lost von 50\% \"uber einen gewissen Zeitraum der erste Alert gesendet werden soll. F\"allt das System komplett aus, soll jedoch ein anderer Alert gesendet werden. Aktuell kennt SmokePing drei verschiedene Alert-Typen: rtt, loss und matcher\footnote{Matcher Alerts sind Plugins, sie erweitern die Alert-Konditionen. Bekannte Matcher sind z.B. AvgRatio, CheckLatency, CheckLoss, Median, MedRatio}. Die Alerts werden nicht durch SmokePing direkt versendet - f\"ur die \"Ubermittlung dieser m\"ussen externe Applikationen oder Skripte vorhanden sein.

\subsubsection{Fazit} \label{sec:systeme-smoke-fazit}
  SmokePing findet seinen Einsatz vor allem in der \"Uberwachung von Netzwerk-Latenzen sowie Latenzen bei verschiedensten Diensten. Durch die geringe Systemlast kann SmokePing sehr gut als nebenl\"aufiges System auf einem existierenden Server eingerichtet werden. Durch das ausgekl\"ugelte Alerting System kann SmokePing zudem auch als Hintergrundprozess laufen und muss nicht dauernd manuell \"uberwacht werden.



\subsection{Zenoss Core} \label{sec:systeme-zenoss} \index{Zenoss Core} \index{Monitoring!Zenoss Core}
  Zenoss Core\cite{zenoss} stellt ein umfangreiches Grundpaket f\"ur Systemmonitoring und Systemmanagement zur Verf\"ugung. Mittels Zenoss Core kann das komplette Netzwerk nachmodelliert und \"uberwacht werden. Es bietet Module zur \"Uberwachung der Erreichbarkeit\footnote{\label{foot:zenoss-monitor}ICMP, SNMP, TCP/IP-Dienste, Windows-Services und Prozesse, Linux/Unix-Prozesse, Nagios-Plugins, ...} und Performance\footnote{\label{foot:zenoss-performance}JMX-Performance von J2EE-Servern, Nagios- und Cacti-Collection Scripts, ...} von Systemen an. Neben einer freien Community-L\"osung kann Zennoss Core auch als Enterprise-Version erworben werden, welche zus\"atzliche Funktionen\footnote{\label{foot:zenoss-enterprise}Virtualisation-, Cloud-, Cisco UCS-Monitoring und Management} sowie Support beinhaltet.

  Die Voraussetzungen f\"ur eine Zenoss-Installation variieren. Basisvoraussetzung ist jedoch \textit{MySQL}, \textit{net-snmp}, \textit{GCJ} und \textit{OpenMP}. F\"ur eine manuelle Installation unter einem Linux/Unix-System werden \textit{GCC/G++}, \textit{GNU build environment}, \textit{SWIG} und \textit{Autoconv} vorausgesetzt. Zus\"atzlich m\"ussen manuell Verzeichnisse und ein Systembenutzer angelegt, sowie ein Init-Skript erstellt werden, welches die Zenoss-Dienste bei einem Systemstart ausf\"uhrt. Bei der Installation \"uber einen Paketmanager entfallen alle diese Schritte und Systemeingriffe.

  Ist das System installiert und gestartet, kann Zenoss Core direkt \"uber den Browser aufgerufen und konfiguriert werden. Die verschiedenen Systeme und Komponenten k\"onnen dabei mittels Plugins, Kommandos, SNMP-OIDs, etc. konfiguriert werden. F\"ur weiterf\"uhrende \"Uberwachungen steht zudem die M\"oglichkeit bereit, ein vorkonfiguriertes Kommando mittels SSH\footnote{\label{foot:zenoss-ssh}F\"ur eine SSH-Verbindung muss sich der Benutzer unter welchem Zenoss Core l\"auft mittels Zertifikat (ohne Passwort) auf dem entfernten System anmelden k\"onnen} auf einem entfernten System auszuf\"uhren sowie die SNMP-OIDs zu erweitern\footnote{\label{foot:zenoss-snmp}F\"ur individualisierte SNMP-OIDs muss sowohl der \"Uberwachungs-Server als auch der entfernte Server angepasst und mit den zus\"atzlichen Informationen ausgestattet werden, was schnell zu einer komplexen Angelegenheit werden kann}.

  Durch verschiedene Benutzergruppen und Berechtigungen kann das System auch f\"ur Kunden ge\"offnet werden, welche zum Beispiel nur ihre eigenen Komponenten einsehen oder auch konfigurieren k\"onnen. Jeder Benutzer, der sich anmeldet, sieht zuerst ein Dashboard, welches die wichtigsten Ereignisse und Alerts aufzeigt. Verschiedene Unterseiten zeigen weitere Details zu den verschiedenen Systemen und Meldungen. Diese sind jedoch f\"ur Laien teilweise nur schwer verst\"andlich und erscheinen einem unge\"ubten Benutzer schnell kryptisch und komplex.

  Das Alerting bei Zenoss Core basiert auf Regeln, welche sehr umfangreich konfiguriert und verschachtelt werden k\"onnen. So k\"onnen \"ahnlich wie bei SmokePing (siehe Kapitel \ref{sec:systeme-smoke}) verschiedene Schweregrade definiert und nacheinander ausgel\"ost werden. Das Alerting bei Zenoss Core basiert ausschliesslich auf SMTP, was bei einem EMail-Serverausfall, respektive dessen "`nicht-Erreichbarkeit"', schwerwiegende Probleme nach sich ziehen kann - Systemausf\"alle k\"onnen nicht mehr gemeldet werden, die Fehler werden also wahrscheinlich nicht bemerkt und nicht behoben.

\subsubsection{Fazit} \label{sec:systeme-zenoss-fazit}
  Zennoss Core wird wohl haupts\"achlich in der Server- und Dienst\"uberwachung eingesetzt. Durch das Alerting und dessen Konfigurationsm\"oglichkeiten kann das System gut ohne visuelle \"Uberwachung eingesetzt werden.


\subsection{Zabbix} \label{sec:systeme-zabbix} \index{Zabbix} \index{Monitoring!Zabbix}
  Zabbix\cite{zabbix} ist eine OpenSource Monitoring L\"osung, welche sowohl Polling\footnote{\label{foot:polling}Polling bedeutet das ein Status explizit angefragt wird, zum Beispiel ICMP/Echo-Ping} wie auch Trapping\footnote{\label{foot:trapping}Trapping heisst, dass Meldungen empfangen werden, welche nicht angefragt worden sind, zum Beispiel SNMP-Traps} zur \"Uberwachung von Netzwerken und Systemen anbietet. Durch zus\"atzliche, auf den Systemen installierte Agenten, k\"onnen neben den standardm\"assig angebotenen Informationen in SNMP-Anfragen noch weitere Daten abgefragt werden. Die \"Uberwachung von TCP/IP-Diensten und ICMP-Pr\"ufungen geh\"oren ebenfalls zu den Grundfunktionalit\"aten von Zabbix. Durch die \texttt{Auto-Discovery}-Funktion\footnote{\label{foot:zabbix-autodisc}Durch Auto-Discovery wird das gesamte Netzwerk automatisch ausgelesen und die verf\"ugbaren Ger\"ate erkannt} kann ein Netzwerk schnell und einfach in Zabbix konfiguriert und ohne grossen Aufwand \"uberwacht werden. Zus\"atzlich zur zentralen Weboberfl\"ache wird auch eine API angeboten, \"uber welche Informationen ausgelesen werden k\"onnen.

  Zabbix ist unterteilt in vier Hauptkomponenten. Der Zabbix-Server stellt die zentrale Stelle dar, welche alle Aktionen und Events sendet, empf\"angt, auswertet und Alerts ausl\"ost. Der Zabbix-Proxy ist ein Meldungs-Puffer, welcher Nachrichten empf\"angt und dem zentralen Zabbix-Server weiterleitet. Die Zabbix-Agents werden auf Systemen installiert, um lokale Ereignisse und Ressourcen zu \"uberwachen und den zentralen Server dar\"uber zu informieren. Das Web-Frontend gew\"ahrt visuellen Zugang zu allen Daten, Graphen und der Konfiguration.

  Das Zabbix Frontend ist in PHP geschrieben und setzt daher einen Webserver mit PHP\footnote{\label{foot:zabbix-php}PHP muss GD, TrueType, BCMath, XML, Session, Socket, MultiByte und MySQL/Oracle/PostgreSQL/SqLite3 unterst\"utzen} voraus. Zus\"atzlich sollte der Server OpenIPMI- und SSH-Unterst\"utzung bieten, damit der komplette Funktionsumfang genutzt werden kann. Bei einer manuellen Installation der Zabbix-Komponenten werden zus\"atzliche Bibliotheken\footnote{\label{foot:zabbix-libs}Build-Utils, GCC/G++ sowie Header und Libraries von MySQL (resp. diejenigen von Oracle, PostgreSQL, SQLite), net-snmp, Iksemel (f\"ur Jabber-Alerts) und Libcurl} vorausgesetzt, welche bei der Installation mittels einem Paket-Manager oder unter Windows nicht gebraucht werden. Nach der Installation m\"ussen noch Services und Startskripte eingerichtet werden um dem Server, den Agenten oder den Proxy automatisch auszuf\"uhren. Die Weboberfl\"ache muss in ein Verzeichnis auf dem Webserver kopiert und mittels einem Browser anschliessend konfiguriert werden.

  Die Grundkonfiguration aller Zabbix-Dienste (Server, Proxy und Agent) wird \"uber eine Konfigurationsdatei gemacht. Die weiterf\"uhrenden Einstellungen werden anschliessend \"uber die zentrale Weboberfl\"ache get\"atigt. Durch vorgefertigte Host-Templates l\"asst sich schnell ein einfaches Monitoring einrichten. Zabbix erlaubt jedoch auch, jeden Event und jede Aktion mit Makros und individuellen Pr\"ufungen zu untersuchen oder neue Datenquellen und Events anzulegen. Dies erlaubt es, ein hoch komplexes und ausgereiftes Monitoring-Netzwerk zu erstellen, welches jedoch den Nachteil hat, dass es oft komplex und nur noch schwer konfigurierbar wird.

  Das Alerting von Zabbix ist \"ahnlich wie das Monitoring: Einfach in der Grundausstattung und sehr stark individualisierbar. Events werden empfangen und entsprechende Aktionen eingeleitet. Die verschiedenen Aktionen k\"onnen die Events dann entweder zur\"uckhalten oder an die n\"achsten Aktionen weiterleiten. So kann, wie zum Beispiel bei SmokePing (siehe Kapitel \ref{sec:systeme-smoke}), ein Event unterschiedlich abgehandelt werden und je nach Ereignis und Zeitpunkt, ein unterschiedlicher Alert versendet werden. Ein Alert kann \"uber SMTP, Jabber, GSM-Module oder individuelle Skripte versendet werden.

\subsubsection{Fazit} \label{sec:systeme-zabbix-fazit}
  Zabbix wird vielfach in gr\"osseren und auch verteilten Netzwerken eingesetzt. Das Einsatzgebiet ist dabei auf Grund der Erweiterbarkeit und der Flexibilit\"at nicht beschr\"ankt. Durch das Alerting und dessen Konfigurationsm\"oglichkeiten kann ein System gut ohne visuelle \"Uberwachung eingesetzt werden. Die hochverf\"ugbare \"Uberwachung scheitert auch bei Zabbix aufgrund des zentralen \"Uberwachungsservers, auch wenn andere Gegebenheiten wie die Agenten darauf ausgelegt zu sein scheinen.


\subsection{Nagios Core} \label{sec:systeme-nagios} \index{Nagios Core} \index{Monitoring!Nagios}
  Nagios Core\cite{nagios} stellt eine zentrale Einheit zur allumfassenden \"Uberwachung von IT-Infrastrukturen bereit. Dabei k\"onnen sowohl Systeme und Applikationen wie auch Dienste mittels Polling und Trapping \"uberwacht werden. Durch die NRPE-Erweiterung\footnote{\label{foot:nagios-nrpe}Die NRPE-Erweiterung ist ein Agent auf dem zu \"uberwachenden System, welche direkt vom Nagios Core Server angesprochen werden kann} k\"onnen auf den zu \"uberwachenden Systemen zus\"atzliche Informationen abgefragt werden. Die F\"ahigkeit, Daten \"uber Agenten zu empfangen, wird von Nagios und vielen anderen verkauft als "`Clustering"' - was es jedoch nicht ist, denn die Agenten sammeln nur Daten und reichen diese an die zentrale Instanz weiter. Durch die grosse Verbreitung und Akzeptanz von Nagios sind viele Erweiterungen, Plugins, Patches und Tutorials im Internet vorhanden. Eine Sammlung mit \"uber 350 Addons, 1700 Plugins und vielem mehr, ist beispielsweise unter \url{http://exchange.nagios.org/} zu finden.

  Nagios setzt nicht wie andere Monitoring-Systeme auf eine Skript-Sprache, sondern ist komplett in C geschrieben. F\"ur die zentrale Web-Oberfl\"ache wird jedoch ein Webserver vorausgesetzt, welcher f\"ur die Erstellung der Graphen die GD2-Bibliothek ben\"otigt. Die Installation ist recht umfassend und setzt einiges an Wissen auf dem jeweiligen Betriebssystem voraus. Zuerst m\"ussen Benutzer und Gruppen angelegt werden und anschliessend der Code, mittels der Angabe des Web-Verzeichnisses, konfiguriert und mittels f\"unf verschiedenen Kommandos\footnote{\label{foot:nagios-compile}\texttt{make all; make install; make install-init; make install-config; make install-commandmode}} kompiliert und installiert werden. Anschliessend m\"ussen alle Beispiel-Konfigurationsdateien kopiert und auf das System angepasst werden. Sind alle Konfigurationen gemacht, muss noch die Web-Oberfl\"ache konfiguriert\footnote{\label{foot:nagios-compile-web}\texttt{make install-webconf}} und mittels einem Passwort\footnote{\label{foot:nagios-compile-pass}\texttt{htpasswd -c /path/to/nagios/etc/htpasswd.users nagiosadmin}} abgesichert sowie der Webserver neu gestartet werden. Sollen noch Plugins in Nagios eingebettet werden, m\"ussen diese ebenfalls zuerst heruntergeladen, entpackt, kompiliert und installiert werden. Schlussendlich k\"onnen die Start-Skripte kopiert, angepasst und dann Nagios Core gestartet werden.

  Die gesamte Konfiguration von Nagios und den zu \"uberwachenden Systemen geschieht \"uber Konfigurationsdateien - eine grafische Konfigurationsm\"oglichkeit via Browser wird standardm\"assig nicht angeboten. Die Konfiguration kann entweder in einer zentralen Datei erfolgen oder \"uber verschiedene Dateien, welche jedoch manuell eingebunden werden m\"ussen. Bevor ein System \"uberwacht werden kann, muss eine Host-Konfiguration\footnote{\label{foot:nagios-cfg-host}define host \{ ... \} - Definition eines Hosts basierend auf einem Host-Template} angelegt werden. Dienste, welche auf einem Host \"uberwacht werden sollen, k\"onnen anschliessend \"uber einen \texttt{service}\footnote{\label{foot:nagios-cfg-service}define service \{ ... \} - Definition eines Service zur \"Uberwachung eines command}-Abschnitt definiert werden. Diese Service-Abschnitte basieren auf Kommandos, welche mittels \texttt{command}-Bl\"ocken\footnote{define command \{ ... \} - Definition eines Kommandos zur \"Uberwachung von SNMP, NPRE, etc.} definiert werden k\"onnen. Nach jeder Anpassung einer Konfiguration sollte zuerst die Konfiguration durch Nagios gepr\"uft\footnote{\label{foot:nagios-config-check}Die Pr\"ufung der Konfiguration geschieht durch das manuelle aufrufen des Nagios-Binaries mit dem Parameter \texttt{-v /path/to/nagios/etc/nagios.cfg}} und anschliessend der Serverdienst neu gestartet werden.

\subsubsection{Fazit} \label{sec:systeme-nagios-fazit}
  Nagios wird vielfach in gr\"osseren und auch verteilten Netzwerken eingesetzt. Das Einsatzgebiet ist dabei aufgrund der Erweiterbarkeit und der Flexibilit\"at sehr gross. Durch das Alerting und dessen Konfigurationsm\"oglichkeiten kann das System gut ohne visuelle \"Uberwachung eingesetzt werden. Die hochverf\"ugbare \"Uberwachung scheitert auch bei Nagios aufgrund des zentralen \"Uberwachungsservers, auch wenn andere Gegebenheiten wie die Agenten darauf ausgelegt sind. Ein zus\"atzlicher Schwachpunkt ist die Konfiguration, denn diese kann durch die verschiedenen Abschnitte und Dateien sehr komplex und un\"ubersichtlich werden.


\subsection{WhatsUp gold} \label{sec:systeme-wug} \index{WhatsUp Gold} \index{Monitoring!WhatsUp Gold}
  WhatsUp Gold\cite{whatsupgold} ist eine allumfassende Netzwerk- und System\"uberwachung, welche sich durch eine einfache Bedienung und Konfiguration auszeichnet. WhatsUp Gold ist ausschliesslich unter Windows mit IIS\footnote{\label{foot:wug-iis}WhatsUp Gold setzt einen IIS-6 oder IIS-7 voraus} lauff\"ahig und setzt einen Microsoft SQL\footnote{\label{foot:wug-sql}Bei den MS-SQL Servern ist zu beachten, dass auch die Gratisversionen zum Einsatz kommen k\"onnen, diese jedoch ein Datenlimit haben}-Server voraus. Weitere Voraussetzungen sind Microsoft Internet Explorer ab Version 7, Microsoft .NET Framework ab 3.51 SP1, Windows Scripting Host ab v5.7 und Microsoft SAPI ab 5.1 f\"ur Text-to-Speech Aktionen.

  Wenn alle Voraussetzungen gegeben sind, wird wie unter Windows \"ublich WhatsUp Gold mittels einem Installer installiert. Die anschliessende, initiale Konfiguration kann \"uber einen Setup-Assistenten gemacht werden. Dabei k\"onnen die grundlegenden Alert-Zeiten und Funktionen definiert sowie Systeme hinzugef\"ugt werden. Die Ger\"ate k\"onnen dabei durch ein HostDiscovery automatisch gesucht oder manuell hinzugef\"ugt werden. Den Ger\"aten k\"onnen anschliessend Rollen\footnote{\label{foot:wug-roles}Rollen sind bei WhatsUp Gold Definitionen, welche den Ger\"atetyp beschreiben. Dies sind zum Beispiel: Webserver, Switch, Mailserver, Drucker, etc.} vergeben werden. Durch die automatische Suche nach Ger\"aten wird ebenfalls automatisch ein Netzwerkplan erstellt.

  Nach der Konfiguration kann der Status des Netzwerkes und der Systeme \"uber die zentrale Weboberfl\"ache betrachtet werden. Diese kann teilweise individualisiert werden, damit die wichtigsten Systeme jederzeit im Auge behalten werden k\"onnen. Durch Anw\"ahlen eines Hosts k\"onnen weitere Details sowie vergangene und aktuelle Alerts und Events eingesehen werden.

  Das Alerting bei WhatsUp Gold folgt einfachen Regeln, welche global konfiguriert werden k\"onnen. Die Regeln beschreiben, was nach welcher Zeit gemacht werden soll. So kann zum Beispiel eingestellt werden, dass beim Bemerken eines Ausfalls zuerst nichts unternommen, wenn der Ausfall \"uber mehr als 5 Minuten andauert ein SMS versendet und wenn der Ausfall l\"anger als 10 Minuten dauert zus\"atzlich noch ein EMail gesendet werden soll. Alerts k\"onnen bei WhatsUp Gold nur \"uber EMail oder auch \"uber angeh\"angte GSM-Module per SMS versendet werden.

\subsubsection{Fazit} \label{sec:systeme-wug-fazit}
  WhatsUp Gold ist ein sehr umfangreiches Paket, welches mit kostenpflichtigen Erweiterungen individuell auf den eigenen Betrieb ausgelegt werden kann. Durch den hohen Bedarf an Ressourcen sowie der Voraussetzung eines IIS inklusive MS-SQL sollte WhatsUp Gold, wenn m\"oglich, auf einem eigenen Server installiert werden. Dem Marketing zufolge ist WhatsUp Gold ein hochverf\"ugbares Monitoring System. Bei genauerer Betrachtung scheitert auch dieses System an einer einzelnen zentralen \"Uberwachungs-Instanz, jedoch mit der M\"oglichkeit von FailOver-Destinationen\footnote{Eine FailOver-Destination ist eine weiter Installation, welche bei einem Ausfall des Hauptsystems zum Einsatz kommt.}.


\subsection{NeDi} \label{sec:systeme-nedi} \index{NeDi} \index{Monitoring!NeDi}
  NeDi\cite{nedi} stellt eine Sammlung von Skripten dar, welche mittels CDP (Cisco Discovery Protocol) und LLDP (Link Layer Discovery Protocol) ein komplettes Netzwerk automatisch untersuchen. Ger\"ate, welche SNMP unterst\"utzen, werden mittels den SNMP-Location-Strings\footnote{\label{foot:nedi-snmp-str}Die SNMP-Location-Strings sollten immer den Aufbau \texttt{Region;Stadt;Geb\"aude;Stockwerk;Raum;[...]} haben} automatisch lokalisiert, auf einem Netzwerkplan eingef\"ugt und gekennzeichnet. Bei Ger\"aten, welche keine SNMP-Unterst\"utzung anbieten, wird versucht anhand der Informationen der CDP/LLDP-Anfragen zu erfahren, um was f\"ur ein Ger\"at es sich handelt. Die Erstellung des Netzwerkplans ist jedoch nicht abh\"angig von der SNMP-F\"ahigkeit. Dieser wird mittels den Daten aus den CDP/LLDP-Anfragen erstellt. Die gefundenen Systeme und Ger\"ate werden in regelm\"assigen Abst\"anden erneuert und k\"onnen \"uber SNMP, SSH oder einfache Telnet-Verbindungen weiter untersucht und \"uberwacht werden. Die gesammelten Daten werden in einer zentralen MySQL-Datenbank gespeichert. Zus\"atzlich kann f\"ur Langzeit-Visualisierungen noch RRDtool mitverwendet werden. NeDi ist so konzipiert, dass die \"Uberwachungs-Skripte, die Weboberfl\"ache und der Datenbankserver auf getrennten Systemen laufen k\"onnen.

  NeDi ist komplett in Perl geschrieben und setzt nur wenige zus\"atzliche Module\footnote{\label{foot:nedi-perl}Net::SNMP, Net::Telnet::Cisco, Algorithm::Diff, DBI, DBD::MySQL, LWP, Net::SSH::Perl, Net::SMTP inklusive libnet} voraus. Die Weboberfl\"ache ist in PHP geschrieben und setzt demzufolge einen Webserver inklusive PHP mit Net-SNMP Unterst\"utzung voraus. Durch die Verwendung von Skript-Sprachen kann NeDi einfach auf einen Server kopiert\footnote{\label{foot:nedi-install}Wird NeDi nicht nach \texttt{/var/nedi/} installiert, m\"ussen in einer PHP-Datei noch Pfade angepasst werden. Das Komplette \texttt{html}-Verzeichnis muss nach der Installation in ein Verzeichnis auf dem Webserver kopiert werden.} und anschliessend konfiguriert werden. Die Datei \texttt{nedi.conf} stellt dabei die zentrale Konfiguration dar. Optional kann zudem eine Liste mit IP-Adressen angelegt werden\footnote{\label{foot:nedi-seed}Die Datei \texttt{seedlist} beinhaltet alle gefundenen IP-Adressen. Initial kann eine Liste eingegeben werden bei welchen NeDi anfangen soll mit dem Polling.}, bei welchen NeDi mit der Suche nach neuen Ger\"aten anfangen soll. Nachdem NeDi initialisiert\footnote{\label{foot:nedi-init}Die Initialisation von NeDi und der Datenbank geschieht mittels \texttt{./nedi.pl -i}} worden ist, kann das Netzwerk nach Ger\"aten durchsucht werden\footnote{\label{foot:nedi-polling}Das Netzwerk kann durch \texttt{./nedi.pl -cob} nach Ger\"aten durchsucht werden}. F\"ur das Monitoring m\"ussen jeweils beim Systemstart noch die zwei Skripte \texttt{moni.pl} und \texttt{syslog.pl} sowie der SNMP-Trap Dienst \texttt{snmptrapd} gestartet werden. Damit NeDi das Netzwerk automatisch nach neuen oder nicht mehr vorhandenen Ger\"aten durchsucht, sollte zudem noch ein Cron-Job\footnote{\label{foot:nedi-cron}Je nach Netzwerk Gr\"osse muss darauf geachtet werden, dass nicht zwei Scan-Prozesse parallel aufgerufen werden.} eingerichtet werden.

  Neben dem Netzwerkplan, welcher eine der zentralen Einheiten von NeDi darstellt, k\"onnen \"uber die Weboberfl\"ache alle Informationen von den Ger\"aten angezeigt werden. Dabei werden je nach vorhandenen Daten, Graphen und Listen mit \"uberwachten Diensten und deren Verf\"ugbarkeit angezeigt. \"Uber das Reporting k\"onnen Daten von Netzwerken, Ger\"aten, Modulen, Interfaces und vielem mehr live zusammengestellt und ausgewertet werden. Bei der Anzeige des Netzwerkplans kann gew\"ahlt werden, welche Bereiche/Gruppen dargestellt werden sollen. Die Gruppen werden dabei aus dem SNMP-Location-Strings extrahiert und zur Auswahl angeboten. Zus\"atzlich k\"onnen noch weitere Filter definiert und Traffic- oder andere Graphen eingeblendet werden (sofern die Ger\"ate diese Informationen anbieten).

\subsubsection{Fazit} \label{sec:systeme-nedi-fazit}
  NeDi wird wohl vielfach zur \"Uberwachung der Verf\"ugbarkeit von Ger\"aten in einem Firmennetzwerk eingesetzt. Zus\"atzlich zu der Verf\"ugbarkeit k\"onnen auch einfache Dienste zus\"atzlich in das Monitoring eingeschlossen werden. NeDi scheint zudem eher f\"ur eine manuelle/visuelle als f\"ur eine autonome \"Uberwachung ausgelegt zu sein.


\subsection{Fazit aller Systeme} \label{sec:systeme-fazit}
  Die meisten Monitoring-Systeme sind auf ein spezielles Gebiet in der Netzwerk\"uberwachung ausgelegt. Sei dies nun die automatische Komponenten-Lokalisierung und \"Uberwachung wie sie NeDi bietet oder die reine Dienst- und Performace\"uberwachung von Cacti und SmokePing.

  Interessante Funktionen bieten fast alle untersuchten Systeme: \textbf{AutoDiscovery} von NeDi und Zabbix, \textbf{Proxys und Agents auf entfernten Systeme} von Zabbix und Nagios, \textbf{Performance-Tests} von SmokePing, \textbf{individuelle SNMP-OIDs} bei Cacti und Zenoss sowie die Verwendung von verschiedenen Netzwerk-Technologien wie \textbf{SNMP(-Traps), Syslog, ICMP, OpenIPMI} etc.

  F\"ur die hochverf\"ugbare \"Uberwachung ist keines der Systeme ausgelegt. Alle Systeme setzten auf eine zentrale Instanz, welche alle umliegenden Systeme \"uberwacht. F\"allt diese zentrale Instanz aus, ist das komplette Netzwerk nicht mehr \"uberwacht. WhatsUp Gold beschreitet diesbez\"uglich eine Vorreiterrolle, denn f\"allt hier die Hauptinstanz aus, \"ubernimmt dessen Funktion eine FailOver-Destination - dieses "`Clustering"' wird auch von einigen Spezial-Systemen angewendet, welche jedoch eher f\"ur Rechenzentren und spezielle Server denn f\"ur Individual\"uberwachungen sind. Zabbix und Nagios setzen diesbez\"uglich auf eine andere Technik, den Proxy. Dieser empf\"angt die Nachrichten von den umliegenden Agenten und Systemen und leitet sie dann an die zentrale Instanz weiter, sobald diese wieder verf\"ugbar ist.

  Keines der untersuchten Systeme kann ohne zus\"atzliche Technik, wie einer FailOver-Anbindung der zentralen Instanz oder des betreibens dieser in einem Cluster, hochverf\"ugbar \"uberwachen. Die meisten Systeme setzen zudem noch voraus, dass verschiedene Services wie ein Webserver, Datenbankserver, etc. funktionst\"uchtig sind. Es wird also genau das vorausgesetzt, was mit einer \"Uberwachungs-Software eigentlich \"uberwacht werden sollte: Eine Lauff\"ahige Infrastruktur und Server. Ist das nicht der Fall, laufen also gewisse Dienste nicht, wird das Netzwerk nicht \"uberwacht und es kann nicht gemeldet oder visuell betrachtet werden, dass etwas nicht l\"auft.

  Bei Firmen mit entsprechendem Budget mag Clustering oder der Einsatz von Hersteller-Abh\"angigen L\"osungen kein Hindernis sein, kleinere Firmen, welche jedoch keinen entsprechenden Etat aufweisen, k\"onnen sich den Einsatz dieser Technik nicht leisten. Dies ist Anlass genug, ein solches System zu planen und durch eine TechDemo dessen Funktionalit\"at und Wirksamkeit zu belegen.

