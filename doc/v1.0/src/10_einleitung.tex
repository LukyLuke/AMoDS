
Die Idee zu einem autonomen \"Uberwachungssystem wurde schon vor einigen Jahren geboren. Die damaligen Systeme, wie auch die aktuellen, basieren alle auf dem Prinzip der zentralisierten \"Uberwachung: Eine zentrale Instanz \"uberwacht alle anderen Systeme und reagiert entsprechend bei einem Ausfall.

Die Problematik, welche sich bei uns in der Firma delight software gmbh immer stellte ist, dass wir viele kleine verteilte Server und Netzwerke \"uberwachen sollten, welche meistens noch durch eine Firewall abgesichert sind. Hinzu kommt, dass die privaten Internetleitungen als nicht sehr stabil und zuverl\"assig angesehen werden konnten. Es kam also immer wieder zu totalen \"Uberwachungs-Ausf\"allen und entsprechenden Fehlalarmen. Basierend auf dieser Tatsache wurde die Idee geboren, dass alle diese Systeme sich doch prinzipiell gegenseitig \"uberwachen k\"onnten. Zus\"atzlich soll das System sich automatisch auf neue Gegebenheiten einstellen k\"onnen, ohne eine aufw\"andige Konfiguration.

Grosse Firmen, welche meistens zwei oder mehrere unterschiedliche Rechenzentren haben, setzen vielfach auf \"Uberwachungssysteme, welche als Cluster betrieben werden k\"onnen. F\"allt ein Rechenzentrum oder System aus, wird automatisch ein anderes daf\"ur eingesetzt. Dabei bemerken die Benutzer vielfach nicht mal, dass ein solcher Wechsel statgefunden hat, da alle Daten redundant auf allen Systemen verteil sind. F\"ur solche interne abgeschirmte Netzwerke sind diese clusterf\"ahigen \"Uberwachungsl\"osungen ideal; Nicht aber wenn viele unterschiedliche Systeme und Komponenten aus verschiedenen Netzwerken mit unterschiedlichen nicht redundanten Anbindungen \"uberwacht werden sollen.

Da in einer kleinen Firma f\"ur solche "`ideologischen"' Projekte meist nur begrenzt Zeit zur verf\"ugung steht, wurde die Idee solange auf Eis gelegt, bis Zeit daf\"ur aufgewendet werden konnte. Die Bachelor-Arbeit hat sich geradezu daf\"ur angeboten, denn nicht nur eine Analyse der aktuellen Systeme, sondern auch eine Machbarkeits-Studie sind f\"ur ein solches Projekt unumg\"anglich.

In einem ersten Schritt werden bestehende \"Uberwachungs-Systeme und deren Funktionsumfang analysiert. Anschliessend wird ein hochverf\"ugbares autonomes \"Uberwachungssystem theoretisch aufgebaut und hergeleitet, sowie im dritten Teil anschliessend als Prototyp/TechDemo ausgearbeitet. Diese Technologie-Demonstration wird einen relativ eingeschr\"ankten Funktionsumfang haben. Es soll damit gezeigt werden, dass und wie ein solches System funktionieren kann.
